\section{Stakeholder Feedback}

  In order to determine the business viability of the proposed ideas, the group drafted a proposal and sent it to multiple stakeholders in an attempt to get some feedback. Their responses are as follows:

  \subsection{Stakeholder 1}

    I think your solution is rather complex. Implementing context/grammar checking and analysing the intent of language statements sounds very difficult. How are you going to go about doing it?

    Why not simply provide a service that will scan the Internet for all traces of your existence through your multiple identities and then automatically remove them. I think that is a sellable service. The Breadcrumb concept has the connotations of a CV service which still requires manual intervention to make things fit. I cannot see this being automated as I think you are hoping for.

  \subsection{Stakeholder 2}

    I gave this a read and found it really interesting. Some of the examples have some relevance to me so I really liked the idea. As someone coming to the business proposal with completely fresh eyes, I'll try to highlight the areas I particularly liked and the areas that need a bit more explanation.

    \subsubsection{Overview and Business Model}

      Firstly, I really liked the name and the explanation for this. I think that you make a really good link to technology and the internet. With regards to the business idea, I liked the idea of having packages for different users. I also liked the pricing structure proposed, along with the add-ons available.

    \subsubsection{Personal Offering}

      I understand that this is geared at someone like me that wants to understand and control their online presence on social media? I think that here you probably need to expand on the selling point. Why would I really need this? There needs to be a reason why this will really benefit an ordinary person like me. For example, if I wanted to improve my presence and performance on social media, maybe the app would help to collate my previous posts which have been most successful or received the most likes so that I know what style of posts are popular and where my profile has been ``positive''.

      With the idea of it creating two lists, I don't understand the difference between the primary and secondary list that the app creates. I don't understand the difference between the two lists or the need for these.

      In addition, I think you could perhaps explain a bit more how it decides what information is significantly positive or negative. Does it scan for trigger words or consider the number of likes?

    \subsubsection{Premier Offering}

      I really liked the celebrity example and the idea of them wanting to maintain a reputation or check for copyright. I think that this example differentiates it from just an advanced search engine. I feel there is danger in the other examples that it is just a search engine. I could be wrong. I think with the football agent situation you need to again tighten up the reasons why they are using it and the reason it would benefit them. Are they using it to find out about what is said on a player or to find information to put together a profile? Again, is this just a search engine? What makes it less of a search engine is the ``sentiment score'' feature. How does it pick up sentiment? Again, is this from trigger words? You might want to explain.

    \subsubsection{Enterprise Offering}

      I like this tier. My sister is setting up her own business so when I think about the app from her perspective, I can see how it would be useful. I think it would be good for a small business to be able to see where people are talking about your brand and the ``sentiment'' of their feedback.

      I think that in this tier, there should be differentiation between small businesses and larger organisations using it as HR screening. The idea of large organisations using this app is huge and a separate tier to small businesses. And, are you suggesting that we should be searching someone's social media when they apply for a job? That's scary!

  \subsection{Stakeholder 3}

    \begin{itemize}
      \item It's an impression management tool which, if I'm correct, aims to alter the perception others have of you based on what can be found on the internet. Could the software consider the presentation of statistical data of regularity at which the deemed positive and negative situations appear to subconsciously make the user more conscious of their online behaviour? I suggest this because if people try hard enough the information can still be found online so to reduce this reduces the risk.
      \item I quite like the name ``Breadcrumb''. Great marketing by the team here. Fairy tales are generally told to children, however, the idea relates a story known by many, from the young to the old, so makes the purpose of the product easy to understand for all and applicable to the majority. I like how it takes present day issue where people are becoming more aware of their digital footprint.
      \item Not sure about following statement: \emph{``Breadcrumb then scans each social media account for content it initially considers to be significantly positive or negative and collates a primary list. Further to this, it trawls the internet to find material that it initially considers to be significantly positive or negative and collates a secondary list''}. It appears to be a very subjective statement. Can the specification as to what this is be initiated by the user or will this be purely general and on the basis of parameters set by the app?
      \item It's not been stated what is then done with the information. What does the group actively do to ensure the better representation of an individual online?
      \item With regards to the Premier offering, how would you ensure that the sources of review were unbiased, relevant and respected?
    \end{itemize}

  \subsection{Stakeholder 4}

    I love the name - seriously cool. I'm not sure what stage in the development you're at, but I'm not entirely sure how the application will allow users to 'be forgotten' from what I'm reading in the document. This may be intentional, but thought it was worth mentioning.

  \subsection{Stakeholder 5}

    \begin{itemize}
      \item Have you thought about scanning LinkedIn profiles for the Enterprise offering?
      \item Will the application be available on desktop? In the example of a HR department using it to screen job applicants this may incur further IT costs if it can only be used on smartphones.
      \item For the enterprise offering, are you able to ensure corporates that their data will be protected? I think this would be important within the investment banking sector as regulators are on everyone's back.
    \end{itemize}

  \subsection{Stakeholder 6}

    It's difficult to understand the landscape that this has been set in… a business plan would need financial projections, market research, an action plan and henceforth, which I assume is not part of the scope of this, so I'll ignore. If that isn't the case, let me know and I'll try to help. Also, there is no mention of any challenges foreseen with the implementation of this – along with how they would be overcome.  This would tell the prospective stakeholders that you are realistic with expectations and have thought through the proposal.

    Also, the style of writing is individual – I personally get frustrated with the marketing spiel, wanting a more direct black and white description, but I doubt the judges are looking for the same style.  I had to read the proposal several times to fully understand the idea.

    Having said that, I like the idea.  I don't understand the need to capture the personal account details, as (I understand) the added value is identifying what is in the public domain, and logging in to a restricted account is only looking at a restricted area.

    I think you need to think about the added value – for example, ability to automatically request removal of information, subscribe to credit checking or police checking, extending the checking to contacts, memberships, directorships – everything that would give confidence that an employer check (both ways, for employee and employer) is comprehensive.

    Full corporations would have teams of people responsible for preserving the company's image – so you should target smaller companies who do not have the resources to do it themselves.

  \subsection{Stakeholder 7}

    \begin{itemize}
      \item Depending on the exact brief, it could be worth adding figures, as a means of providing rough estimates over the commerciality of your app. If the brief said short business plan, highlighting commerciality could score you extra brownie points. So if possible, please expand upon the potential revenue generated by each tiered offering. However, if the brief simply says create an app, then ignore the commerical aspect. Furthermore, if you do need to write a business plan and you're worried about word limit, simply make a table outlining your estimates, and then stick it in as an image.
      \item The brief states, that your app plan should cover the perspectives from the view of different stakeholders. I don't feel your current plan directly addresses these stakeholders. I advise that you include a stakeholder perspective section in your finalised business plan. For example:
      \begin{itemize}
        \item For the individual, the service provides a better understanding of their digital footprint
        \item For the citizen, the service collates public records on them, and exercises their right to be forgotten if negative posts keep coming up
        \item For the employee, the service improves and cleans up their digital footprint in order to make them more employable
      \end{itemize}
    \end{itemize}

  \subsection{Stakeholder 8}
	
	I don't know what my rights within this context are at all. Can there be an educational element built into the the app? 
	
	I guess I can see myself using the app to track and monitor my information online. But can I actually take any actions to improve my profile as it stands? For me to take this seriously, the app needs to either have that level of functionality or provide me with the information needed to do it myself.
	
	\subsection{Stakeholder 9}
	
	Great idea. There is a gap in the market for this.
	
	A few positives:
      \begin{itemize}
        \item Freemium business model is a good way to drive sales and promote the business
        \item Interface is simple, clear and easy to understand; works well as a mobile interface and can later be developed to fit a web interface
        \item Good that the service targets more than one audience
      \end{itemize}
	
	Areas for improvement:
      \begin{itemize}
        \item Clarify whether the app retains any information for future use
        \item Interface is similar to Tinder... how do you undo a \emph{``swipe''}? If so, does it come with an extra charge?
        \item Scrap the \emph{``sand at the beach''} marketing pull as it does not have much relevance to the idea
      \end{itemize}
		
		\subsection{Stakeholder 10}
	
	      \begin{itemize}
        \item What guarantee do you offer or who has signed your application off as a trusted source for adding such personal information and social media accounts? Will people know that I have even used Breadcrumb? I might be embarrassed if they did know I was using it to clean up my past history as it may be stuff I am not proud of.
        \item How can the app tell if something is positive or negative? Does it take into account the audience that the message is targeted at. How would it classify the following status updates: \emph{``I was on the beers last week got wrecked but had a good time''} versus \emph{``I failed my exam have to spend another year at university...''}
        \item Which market are you targeting? Have you got some case study examples as to why people would want this? It may only appeal to those who have a bad past and need to tidy up their act or want to remove elements of their history online?
        \item Under the Personal offering section, you mentioned giving rewards to the user that will motivate them to keep using the app. What rewards exactly? Barclays Premier League tickets? A raffle entered into a competition to meet Bill Gates or have dinner with a super model? You need to clarify this.
        \item As for the Premier offering, sounds like you have got a good market to target. Think about American sports as well. There is potential to expand the service worldwide if you provided the service to some world class sportsmen such as Wayne Rooney who may not want to see a \textit{boxing match} on social media.
      \end{itemize}
