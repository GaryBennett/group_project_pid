\section{Literature Review}

The concept of the \emph{``right to be forgotten''} is based on the European ideal of an individual determining the development of their own life in an autonomous manner, without being periodically stigmatised as a consequence of a specific action performed in the past, especially when these events do not have any relationship with the contemporary context~\parencite{art:eu_forgotten}. However, does the \emph{``right to be forgotten''} really have a sound basis?~\textcite{web:no_right_forgotten} argues that the right to being forgotten is a figment of our imagination, describing it as an antisocial, nihilist act which could eventually signify the degradation of our power to act in the world.~\textcite{web:foggy_thinking} extends this argument by asking the difficult question: who should be responsible for what should be remembered or forgotten? For example, if Italian courts decide that Italian murderers should be able to delete all references to their online convictions after a period of time, would this Italian standard apply to the entire Web or only be applied to *.it domains? Under the ruling in Article 12 of the Directive 95/46/EC, Europeans who feel that they are being misrepresented by search results that are no longer accurate or irrelevant can ask Google to de-link the material. If the request was approved, the information would remain online at the original site, but no longer come up under certain search engine queries~\parencite{web:right_to_be_forgotten}. 

It can be argued that certain ideas pertaining to the \emph{``right to be forgotten''} are synonymous with those pertaining to self-presentation and self-disclosure. \textcite{book:self_presentation} ideas on self-presentation asserts the ways in which an individual may partake in strategic activities \emph{``to convey an impression others which it is in their best interests to convey''}. Online self-presentation is more malleable and subject to self-censorship, allowing individuals to express or make salient multiple aspects of their identity. Consequently under certain conditions, individuals may wish to express themselves more openly and honestly than in face to face contexts~\parencite{art:manage_impress} or engage in misrepresentation.

Within the assigned context, the group will need to consider incorporating impression management features in order to create a marketable commercial solution. Impression management can be defined\emph{``as the goal-directed activity of controlling information about a person, object, entity, idea or event''}~\parencite{art:appearing_competent}. As discussed above, the group considers impression management to be an expansion of the \emph{``right to be forgotten''} concept, as it presents an opportunity for individuals and corporates to correct any misrepresentations of themselves online. Relationships are increasingly initiated and maintained online, providing individuals and corporates with an opportunity to create strategic images of themselves for social purposes~\parencite{art:olm}.

Moreover, according to~\textcite{rep:LERC} it creates opportunities for data extracted from these sources to be used in opinion mining and sentiment analysis. Sentiment analysis can be defined as \emph{``the computational study of opinions, sentiments and emotions expressed in texts''}~\parencite{inbook:lang_proc_sec}. For example, during an election campaign, political parties may be interested to know if people support their candidate or not by analysing messages written across multiple social media networks in order to extract an aggregated sentiment which can be used to support decision-making. However in this case, complications may arise as the disclosure domain is boundless, and as a result all messages will need to be filtered as they can refer to any subject~\parencite{art:computers_human_behaviour}.
